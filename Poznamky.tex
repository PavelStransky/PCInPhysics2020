\documentclass[a4paper,11pt,twoside]{article}
\usepackage[utf8]{inputenc}	%% Text coding
\usepackage[czech]{babel}
\usepackage{epsfig,subfigure}
\usepackage{amsfonts,amsmath,amssymb}
\usepackage{graphicx}
\usepackage[unicode]{hyperref}
\usepackage{indentfirst}
\usepackage{fancyhdr}
\usepackage{xifthen}
\usepackage{amsthm}

\hypersetup{
	pdftitle={Využití počítačů ve fyzice},
	pdfauthor={Pavel Stránský},
	pdffitwindow=true,
	colorlinks=true,
	urlcolor=cyan,            %barva textu pri tisku
	linkcolor=red,
	citecolor=green,
	filecolor=magenta
}

% Velikost stránky
\addtolength{\topmargin}{-1.5cm} %\addtolength{\textheight}{-10cm}
\addtolength{\textwidth}{4cm} \addtolength{\textheight}{4cm} % Šířka a výška textu
\addtolength{\voffset}{-0.5cm} % Horní okraj
\addtolength{\hoffset}{-2cm}
\setlength{\headheight}{15pt}

\pagestyle{fancy}

\def\vector#1{\boldsymbol{#1}}								% Vector
\renewcommand{\d}{\mathrm{d}}
\newcommand{\derivative}[3][]{\ifthenelse{\isempty{#1}}	    % Normal derivative
	{\frac{\d{#2}}{\d{#3}}}
	{\frac{\d^{#1}{#2}}{\d{#3}^{#1}}}
}
%\def\example{\textbf{\textit{Příklad:}}}
%\def\example{\subsubsection*{Příklad:}}
\def\makematrix#1{\begin{pmatrix}#1\end{pmatrix}}       % Matrix

%\theoremstyle{definition}

\begin{document}
\newtheorem{example}{Příklad}[section]
\newtheorem{task}{Úkol}[section]

\title{Zápisky k předmětu Využití počítačů ve fyzice}
\date{\today}
\author{Pavel Stránský}

\maketitle

\section{Obyčejné diferenciální rovnice}
    Každou obyčejnou diferenciální rovnici $n$-tého řádu lineární v nejvyšší derivaci lze převést na soustavu
    $n$ obyčejných diferenciálních rovnic prvního řádu ve tvaru
    \begin{equation}\label{eq:ODR}
        \derivative{\vector{x}}{t}=\vector{f}(\vector{x},t),
    \end{equation}
    kde $\vector{x}=\vector{x}(t)$ je vektor hledaných funkcí.

\begin{example}
    Pohybovou rovnici
    \begin{equation}
        Ma=F(y),
    \end{equation}
    kde $M$ je hmotnost pohybujícího se tělesa, $y=y(t)$ jeho poloha a $a=a(t)=\d^{2}y/\d t^{2}$ převedeme
    na dvě diferenciální rovnice prvního řádu triviálním zavedením rychlosti $v=v(t)=\d y/\d t$:
    \begin{equation}
        \derivative{}{t}\makematrix{y \\ v}=\makematrix{v \\ \frac{1}{M}F(y)},
    \end{equation}
    tj. vektor funkce pravých stran podle rovnice~\eqref{eq:ODR} je
    \begin{equation}
        \vector{f}(\vector{x},t)=\makematrix{v \\ \frac{1}{M}F(y)}
    \end{equation}
    kde $\vector{x}\equiv(y,v)$.
\end{example}

\begin{example}
    Pohybová rovnice pro harmonický oscilátor (matematické kyvadlo s malou výchylkou) při volbě jednotek $M=\Omega=1$, kde $M$ je hmotnost kmitající částice a $\Omega$ její rychlost, zní
    \begin{align}
        a&=\derivative[2]{y}{t}=-y &&\Longleftrightarrow &
        \derivative{}{t}\makematrix{y \\ v}&=\makematrix{v \\ t}
    \end{align}

\end{example}

\begin{task}
    Převeďte na soustavu obyčejných diferenciálních rovnic prvního řádu rovnici třetího řádu pro Hiemenzův tok
    \begin{equation}
        y'''+yy''-y'^{2}+1=0.
    \end{equation}
\end{task}

\subsection{Diferenciální rovnice prvního řádu}
    Drtivá většina knihoven a algoritmů pro integraci diferenciálních rovnic počítá s rovnicemi ve tvaru~\eqref{eq:ODR}.
    Zde se omezíme na jednu rovnici
    \begin{equation}
        \derivative{y}{t}=f(y,t),
    \end{equation}
    přičemž rozšíření na soustavu je triviální: místo skalárů $y$ a $f$ vezmeme vektory.

    Řešení diferenciální rovnice spočívá v nahrazení infinitezimálních přírůstků přírůstky konečnými:
    \begin{equation}\label{eq:Difference}
        \frac{\Delta y}{\Delta t}=\phi(y,t)
    \end{equation}
    kde $\phi$ je funkce, která udává směr, podél kterého se při numerickém řešení vydáme.
    Volbá této funkce je klíčová a záleží na ní, jak přesné řešení dostaneme a jak rychle ho dostaneme.

    \subsubsection{Pár důležitých pojmů}
    \begin{itemize}
        \item {\bf Jednokrokové algoritmy:}
        Algoritmy, které výpočtu následujícího kroku hodnoty funkce $y_{i+1}$ vyžadují znalost pouze aktuálního kroku $y_{i}$.
        Rozepsáním~\eqref{eq:Difference} dostaneme
        \begin{equation}
            \boxed{
                y_{i+1}=y_{i}+\underbrace{\phi(y_{i},t)}_{\phi_{i}}\Delta t
            },
        \end{equation}
        přičemž počáteční hodnota $y_{0}$ je dána počáteční podmínkou.
        My se omezíme pouze na tyto algoritmy.

        \item {\bf Lokální diskretizační chyba:}
        \begin{equation}
            \mathcal{L}=y(t+\Delta t)-y(t)-\phi(y(t),t)\Delta t,
        \end{equation}        
        kde $y(t)$ udává přesné řešení v čase $t$.

        \item {\bf Akumulovaná diskretizační chyba:}
        \begin{equation}
            \epsilon_{i}=y_{i}-y(t_{i})
        \end{equation}

        \item {\bf Řád metody:} 
        Metoda je $p$-tého řádu, pokud
        \begin{equation}\label{eq:MethodOrder}
            L(\Delta t)=\mathcal{O}(\Delta t^{p+1})
        \end{equation}

        \item {\bf Symplektické algoritmy:}
        Speciální algoritmy navržené pro řešení pohybových diferenciálních rovnic.
        Od běžných algoritmů je odlišuje to, že zachovávají objem fázového prostoru, a tedy i energii (zatímco u obecných algoritmů energie s integračním časem roste).
        V praxi se ze symplektických algoritmů používá pouze Verletův algoritmus~\ref{sec:Verlet}.

        \item {\bf Kontrola chyby řešení:}
        Chybu numerického řešení diferenciální rovnice lze zmenšit 1) menším krokem, 2) lepší metodou (metodou vyššího řádu). 
        Menší krok však znamená vyšší výpočetní čas.
        Sofistikované metody proto průběžně mění velikost kroku: když se funkce mění pomalu, krok prodlouží, když se mění rychle, krok zkrátí (tzv. {\bf metody s adaptivním krokem}).
        Tím se docílí vysoké přesnosti při co nejmenším výpočetním čase.

    \end{itemize}

    \subsubsection{Eulerova metoda 1. řádu}
        \begin{equation}\label{eq:Euler1}
            \phi_{i}=f(y_{i},t_{i}),
        \end{equation}
        tj. krok do $y_{i+1}$ děláme vždy ve směru tečny v bodě $y_{i}$.

        \begin{itemize}
            \item Nejjednodušší metoda integrace diferenciálních rovnic.
            \item Chyba je obrovská, k dosažení přesných hodnot je potřeba velmi malého kroku, což znamená dlouhý výpočetní čas.
        \end{itemize}

    \subsubsection{Eulerova metoda 2. řádu}
        \begin{align}\label{eq:Euler2a}
            k_{1}&=f(y_{i},t_{i})\nonumber\\
            k_{2}&=f\left(y_{i}+k_{1}\Delta t,t+\Delta t\right)\\
            \phi_{i}&=\frac{1}{2}\left(k_{1}+k_{2}\right),\nonumber
        \end{align}
        tj. uděláme jednoduchý Eulerův krok ve směru $k_{1}$, spočítáme derivaci $k_{2}$ po tomto kroku a vyrazíme z bodu $y_{i}$ ve směru, který je průměrem obou směrů (doporučuji si nakreslit obrázek).

        Ekvivalentní je udělat \uv{Eulerův půlkrok} a vyrazit z bodu $y_{i}$ ve směru derivace spočtené po tomto půlkroku:
        \begin{align}\label{eq:Euler2b}
            k'_{1}&=f(y_{i},t_{i})\nonumber\\
            k'_{2}&=f\left(y_{i}+k'_{1}\frac{\Delta t}{2},t+\frac{\Delta t}{2}\right)\\
            \phi_{i}&=k'_{2}\nonumber
        \end{align}

    \subsection{Runge-Kuttova metoda 4. řádu}
        \begin{align}\label{eq:RungeKutta}
            k_{1}&=f(y_{i},t_{i})\nonumber\\
            k_{2}&=f\left(y_{i}+k_{1}\frac{\Delta t}{2},t+\frac{\Delta t}{2}\right)\nonumber\\
            k_{3}&=f\left(y_{i}+k_{2}\frac{\Delta t}{2},t+\frac{\Delta t}{2}\right)\\
            k_{4}&=f\left(y_{i}+k_{3}\Delta t,t+\Delta t\right)\nonumber\\
            \phi_{i}&=\frac{1}{6}\left(k_{1}+2k_{2}+2k_{3}+k_{4}\right)\nonumber
        \end{align}
        
        \begin{itemize}
            \item Jedna z nejčastěji používaných metod.
            \item Vysoká rychlost a přesnost při relativní jednoduchosti.
            \item Existují i Runge-Kuttovy metody vyššího řádu $p$, avšak vyžadují výpočet více než $p$ dílčích derivací $k_{j}$.
            Obecně platí, že metoda řádu $p\leq4$ vyžaduje $p$ derivací, metoda řádu $5\leq<p\leq7$ vyžaduje $p+1$ derivací a metoda řádu $p=8,9$ vyřaduje $p+2$ derivací.
        \end{itemize}

    \subsection{Verletova metoda}\label{sec:Verlet}
        Pro rovnici 2. řádu ve tvaru (pohybovou rovnici)
        \begin{equation}\label{eq:EM}
            M\derivative[2]{y}{t}=F(y),
        \end{equation}
        kde $M$ je hmotnost pohybující se částice a $F$ síla, která na ni působí.
        Algoritmus je
        \begin{align}
            y_{i+1}&=y_{i}+v_{i}\Delta t+\frac{1}{2}a_{i}\Delta t^{2},\nonumber\\
            v_{i+1}&=v_{i}+\frac{1}{2}\left(a_{i+1}+a_{i}\right)\Delta t,
        \end{align}
        kde $a_{i}\equiv F(y_{i})/M$.

        \begin{itemize}
            \item Symplektický algoritmus, tj. algoritmus zachovávající energii (pokud systém popsaný rovnicí~\eqref{eq:EM} energii zachovává).
            \item Užívá se nejčastěji v molekulární dynamice k simulaci pohybu velkého množství vzájemně interagujících částic.
            \item Řád této metody je $p=2$. Symplektické algoritmy s vyšším řádem existují, avšak v praxi se nepoužívají.
        \end{itemize}

    \begin{task}
        Naprogramujte Eulerovu metodu 1. a 2. řádu\footnote{Tyto metody jsme již naprogramovali na cvičení minulý týden.}, Runge-Kuttovu metodu a Verletovu metodu a vyřešte diferenciální rovnici harmonického oscilátoru
        \begin{equation}
            \derivative[2]{y}{t}=-y
        \end{equation}
        s počátečními podmínkami $y_{0}=0$, $y'_{0}\equiv v_{0}=1$ (funkce $\sin t$).
        Časový krok ponechte jako volný parametr.
        Nakreslete grafy řešení $y(t)$ a grafy energie $E(t)$ pro rozdílné hodnoty integračních kroků, například $\Delta t=0.01$ a $\Delta t=0.1$ pro čas $t\in\langle0;30\rangle$.
        Energie harmonického oscilátoru je dána vzorcem
        \begin{equation}
            E=\frac{1}{2}\left(y^{2}+v^{2}\right).
        \end{equation}
        Přesvědčte se, že jediná Verletova metoda skutečně zachovává energii. 
        Pro ostatní metody energie roste.
    \end{task}

    \begin{task}
        Rozšiřte kód tak, aby počítal kumulovanou kvadratickou chybu
        \begin{equation}
            \mathcal{E}=\sqrt{\sum_{i}\left(y_{i}-\sin t_{i}\right)^{2}}
        \end{equation}
        a nakreslete závislost $E(\Delta t)$ pro $\Delta t\in\langle0.002;0.1\rangle$ a pro různé metody.
        Jelikož očekáváme mocninnou závislost dle~\eqref{eq:MethodOrder}, kde exponent je tím větší, čím větší je řád metody, je výhodné graf $E(\Delta t)$ kreslit v log-log měřítku.
        V Pythonu použijete místo \textnormal{\texttt{plot(...)}} funkci \textnormal{\texttt{loglog(...)}} z knihovny \textnormal{\texttt{matplotlib.pyplot}}.
    \end{task}

    \begin{task}
        Eulerovu metodu 1. řádu lze pro harmonický oscilátor vylepšit následující záměnou:
        \begin{align}
            &\begin{matrix}
                y_{i+1}=y_{i}+v_{i}\Delta t \\
                v_{i+1}=v_{i}-y_{i}\Delta t 
            \end{matrix}
            &&\longrightarrow
            &\begin{matrix}
                y_{i+1}=y_{i}+v_{i}\Delta t \\
                v_{i+1}=v_{i}-y_{i+1}\Delta t 
            \end{matrix}
        \end{align}
        (vypočítáme $y_{i+1}$ a tuto hodnotu použijeme namísto hodnoty $y_{i}$ pro výpočet rychlosti $v_{i+1}$).
        Naprogramujte tuto metodu u ukažte, že pro harmonický oscilátor se jedná o metodu 2. řádu.
        Využijte srovnání v grafu z předchozí úlohy.
    \end{task}

    \begin{task}
        Využijte hotové kódy a pohrajte si s řešením rovnice pro klesající exponenciálu
        \begin{equation}
            \derivative[2]{y}{t}=y
        \end{equation}
        s počátečními podmínkami $y_{0}=1$, $y'_{0}=-1$.
        Přesvědčte se, že Verletova metoda a vylepšená Eulerova metoda z posledního bodu jsou nestabilní --- pro tuto rovnici v relativně krátkém čase začnou řešení exponenciálně divergovat.
    \end{task}
\end{document}
